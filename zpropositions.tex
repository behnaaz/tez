\documentclass[runnngheads]{book}
\usepackage{graphicx}
\usepackage[utf8x]{inputenc}
\usepackage[T1]{fontenc}
\usepackage{wrapfig,lipsum,booktabs}
\usepackage{placeins}
\usepackage{amssymb}
\usepackage{upgreek}
\usepackage{longtable}
\usepackage{tikz}
\usepackage{reo-tikz}
\usetikzlibrary{shapes.geometric}
\usepackage[linesnumbered,lined]{algorithm2e}
\usepackage{pgfplots}
\usepackage{titling}
\usepackage{authblk}
\usepackage{makecell}
\usepackage{amsmath}
\usepackage[colorlinks=true, allcolors=blue]{hyperref}
\usepackage{array}
\usepackage{floatrow}
\floatsetup[table]{capposition=top}
\usepackage{booktabs}
\usepackage{pdflscape}
\usepackage{longtable}
\usepackage{etex}
\usepackage{lipsum}
\usepackage[T1]{fontenc}
\usepackage{amsmath}
\usepackage{caption}
\usepackage{subcaption}
\usepackage{tikz}
\usepackage{reo-tikz}
\usepackage{tikz-bpmn}
\usepackage{graphicx}
\usepackage{multirow}
\usepackage{upgreek}
\usepackage{romanbar}
\usepackage{tabularx}
\usepackage{color}
\usepackage{placeins}
\usepackage{xcolor}
\usepackage{textcomp}
\usepackage{dsfont}
\usepackage{amsbsy}
\usepackage{amstext}
\usepackage{amscd}
\usepackage{amsxtra}
\usepackage{pgf}
\usepackage{amsfonts, lacromay}
\usetikzlibrary{chains, fit, shapes,matrix,decorations.shapes}
\usepackage{amssymb}
\usepackage{array}
\usepackage{syntax}
\usepackage[english]{babel}
\usetikzlibrary{shapes.multipart}
\usetikzlibrary{shapes.geometric,positioning,calc,arrows,snakes}
\usepackage{chngcntr}
\usetikzlibrary{arrows,shadows} % for pgf-umlsd
\usepackage{stmaryrd} %cross product
\usepackage{url}
\usepackage{verbatim}
\usepackage{caption}
\usepackage{listings}
\usepackage{etoolbox}
\floatstyle{plain}
\floatname{program}{Listing}
\usepackage{mathrsfs}
\usepackage[underline=true,rounded corners=false]{pgf-umlsd}
\usetikzlibrary{arrows, decorations.markings}
\usetikzlibrary{shapes.geometric,calc}
\usepackage{tikz}
\usetikzlibrary{arrows, shapes, trees, positioning, decorations.markings, patterns} 
\usepackage{bpmn-events}  
\usepackage{bpmn-gateways}
\usepackage{fancyvrb}
\usepackage{multirow}
\usepackage{caption}
\usepackage{array}
\usepackage{tikz,amsmath, amssymb,bm,color}
\usetikzlibrary{circuits.logic.US,circuits.logic.IEC,fit}
\newcommand\addvmargin[1]{
  \node[fit=(current bounding box),inner ysep=#1,inner xsep=0]{};
}
\newtheorem{BehAxiom}{Axiom}[section]
\newtheorem{BehExample}{Example}[section]
\newenvironment{proof}[1][Proof]{\begin{trivlist}
\item[\hskip \labelsep {\bfseries #1}]}{\end{trivlist}}
\newcounter{examplecounter}
\newenvironment{BehExample3}{%\begin{quote}%
\noindent
 %   \refstepcounter{examplecounter}%
  \textbf{Example \thechapter-\arabic{examplecounter}}%
  \quad
}{%
}
\lstset{
  basicstyle=\ttfamily,
  columns=fullflexible,
  frame=single,
  breaklines=true,
  postbreak=\mbox{\textcolor{red}{$\hookrightarrow$}\space},
}
\def\@makechapterhead#1{%
  \vspace*{50\p@}%                                 % Insert 50pt (vertical) space
  {\parindent \z@ \raggedright \normalfont         % No paragraph indent, ragged right
    \ifnum \c@secnumdepth >\m@ne                   % If you should number chapters
      \if@mainmatter                               % ... and you're in \mainmatter
        \huge\bfseries \@chapapp\space \thechapter % huge, bold, Chapter + number
        \par\nobreak                               % paragraph break without page break
        \vskip 20\p@                               % Insert 20pt (vertical) space
      \fi
    \fi
    \interlinepenalty\@M                           % Penalty
    \Huge \bfseries #1\par\nobreak                 % Huge, bold chapter title
    \vskip 40\p@                                   % Insert 40pt (vertical) space
  }}

\begin{document}
\thispagestyle{empty}
\title{Summary}
%\maketitle
%\newpage
{\textbf{\Large{Summary}}}
\vspace*{1cm}
\\

 Business process management is an operational management 
approach that focuses on improving business processes. Business processes, i.e., collections of important activities in an organization, are 
represented in the form of a workflow, an orchestrated and repeatable 
pattern of activities amenable to automated analysis and control. 

Business Process Model and Notation (BPMN) has become the de-facto standard for business processes diagrams. In order to provide tools support to analyze the behavior of a BPMN model, in this dissertation, we present a mapping of BPMN models to Reo networks. The Reo coordination language is an exogenous coordination language that
realizes the coordination patterns in terms of its complex networks, that are built out of simple
primitives called channels.  The mapping of BPMN to Reo is implemented as a plugin in the Reo analysis tool-set in a model-driven paradigm. Our mapping covers not only basic BPMN constructs but also advanced structures such as BPMN transactions. 

Reo is easily extensible to support more advanced process models by defining new channels. 
However, the open-ended nature of Reo channels makes it necessary to extend the formal semantics of Reo in order to include some new concepts. 

Several dozen variations of semantic models for Reo have been proposed that vary from rather simple that cover basic Reo behavior to more complex models that capture specific behavioral aspects, e.g., context-sensitivity.  In some of these semantic models, computing the overall semantics of a system given  semantics for its parts is computationally expensive. This hampers using the language for analyzing large real-world business processes. 

In this dissertation, we present a constraint-based framework, which unifies various formal semantics of Reo. In this framework, the behavior of a Reo network is described using constraints. The constraint-based nature of our approach allows  the simultaneous coexistence of several semantics in a simple fashion. The behavior of a Reo network is determined by the solutions to these constraints. Since any solution should satisfy all the encoded formal semantics, the framework eliminated any inconsistent behavior between the Reo formal semantics.

Another advantage of our proposed constraint-based approach compared to the existing approaches of calculating formal semantics of Reo is its efficiency due to efficient constraint solving methods and optimization techniques that are used in the off-the-shelf constraint solvers. We support this claim with a case study.

Among the behavioral aspects required to model a business process is {priority}. The notion of priority is necessary for modeling behaviors such as transaction and exception handling, where the data flow representing the error or exception should interrupt the normal flow. 

In this dissertation, we present an alternative approach to model priority in Reo by extending our constraint-based framework with priority-aware premises. Further, we extend our priority-aware formal model to support not only a binary notion of priority, but also numeric priorities. 


%\newpage
%{\textbf{\Large{Samenvatting (Dutch Summary)}}}
\vspace*{1cm}
\\

Bedrijfsprocesbeheer is een operationele managementaanpak die zich richt op het verbeteren van bedrijfsprocessen. Bedrijfsprocessen, d.w.z. verzamelingen van belangrijke activiteiten in een organisatie, worden weergegeven in de vorm van een workflow, een georkestreerd en herhalend patroon van activiteiten die geschikt zijn voor geautomatiseerde analyse en controle. 

Business Process Model and Notation (BPMN) is de algemene standaard geworden voor bedrijfsprocesdiagrammen. 
Om ondersteunening in het analyseren van het gedrag van een BPMN-model,
presenteren we in dit proefschrift een vertaling van BPMN-modellen naar Reo-netwerken. De Reo-coördinatietaal is een exogene coördinatietaal die de coördinatiepatronen opnieuw benoemt in termen van complexe netwerken, die zijn opgebouwd uit simpele primitieven genaamd channels. De vertaling van BPMN naar Reo is geïmplementeerd als een Reo-analysetool in een modelgedreven paradigma. Onze vertaling omvat niet alleen standaard BPMN-constructies, maar ook geavanceerde structuren zoals BPMN-transacties. 

Reo is eenvoudig uitbreidbaar om meer geavanceerde procesmodellen te ondersteunen door het definiëren van nieuwe channels. Het flexibele karakter van Reo-kanalen maakt het echter noodzakelijk om de formele semantiek van Reo uit te breiden met een aantal nieuwe concepten. Verschillende variaties van semantische modellen voor Reo voorgesteld, varierend van vrij eenvoudig en die betrekking hebben op het basisgedrag van Reo, tot meer complexe modellen die specifieke gedragsaspecten vast leggen, bijvoorbeeld contextgevoeligheid. 

In vele van deze semantische modellen is de berekening van de algehele semantiek van het systeem gegeven de semantiek van zijn onderdelen is rekenkundig duur. Dit bemoeilijkt het gebruik van de taal voor het analyseren van grote bedrijfsprocessen. In dit proefschrift presenteren we een op constraint-based framework dat verschillende formele semantiek van Reo. 

In dit framework wordt het gedrag van een Reonetwork beschreven met het behulp van constraints. Onze constraint-based framework aanpak maakt gelijktijdige bestaan van verschillende semantiek op een eenvoudige manier mogelijk. Het gedrag van een Reo-netwerk wordt bepaald door de oplossingen voor deze constraints. Aangezien elke oplossing zou moeten voldoen aan alle gecodeerde formele semantiek, elimineerde het elk inconsistent gedrag tussen de Reo-formele semantieken. 

Een ander voordeel van onze voorgestelde constraint-based benadering vergeleken met de bestaande benaderingen van het berekenen van de formele semantiek van Reo is de efficiëntie door efficiënte constraint-solving methoden en optimalisatietechnieken die worden gebruikt in de off-the-shelf constraint-solvers. We ondersteunen deze bewering met een casestudy. 

Onder de gedragsaspecten die vereist zijn om een bedrijfsproces te modelleren, is prioriteit. Prioriteit is nodig voor het modelleren van gedrag zoals transactions en exception handling, waarbij de dataflow die de error of exception representeert de normale flow moet onderbreken. 

In dit proefschrift presenteren we een alternatieve benadering om prioriteit te modelen in Reo door onze framework uit te breiden met prioriteit. Bovendien breiden we ons prioriteitsbewust formele model uit om niet alleen binaire prioriteit, maar ook numerieke prioriteiten te ondersteunen.


%\newpage
\begin{center}{\large{Propositions}}\end{center}
\begin{center}accompanying the thesis\end{center}

\vspace*{.5cm}
\begin{center}{\Large{Constraint-Based Analysis of Business Process Models}}\end{center}

\begin{center}Behnaz Changizi\end{center}
\vspace*{1cm}
\begin{enumerate}
\item The constraint-based nature of our approach
allows simultaneous coexistence of several semantics in a simple fashion. [Chapter 1]
%\item BPMN bridges the gap between visualization of the business
%processes and their actual implementation by providing an understandable notation
%for both business stakeholders and technical experts. [Chapter 2]
%\item Writing and maintaining glue code is a tedious task, especially in complex
%systems wherein the size and rigidity of the glue code tend to increase over
%time. This makes these systems hard to modify and maintain. [Chapter 3]
%\item A benefit of employing coordination languages in general and Reo in particular
%is that they express the coordination patterns explicitly and separate them
%from the computational part of the code. [Chapter 4] 
\item By using ATL we benefit from the power of separation of concerns and focus
only on the required mapping rules, rather than matching patterns on the source
models and execution of the rules. [Chapter 5]
\item Our framework eliminates the result of expressiveness gap among Reo formal
semantics by incorporating more than one semantics in deriving the behavior
of a Reo connector. [Chapter 6]
\item Rather than implementing the highly time- and memory-demanding
custom-made algorithms to generate Reo formal semantics, we use the efficient
SAT-solvers and computer algebra systems to solve constraints whose solutions
are equivalent to these models. [Chapter 6]
\item When an end with innate
priority connects to another end that has no priority, the new end will obtain
acquired priority. [Chapter 7]
\item Each of these formal semantics can be viewed as a means to constrain the
range of possible behaviors that is expressed in terms of I/O operations through
the nodes to those of allowed by the semantics. [Chapter 8]
\item The beauty of mathematics only shows itself to more patient followers.
\item Every culture has something to be ashamed of, but every culture also has the right to change, to challenge negative traditions, and to create new ones.
\item I find that through the study of women, you get to the heart - the truth - of the culture.
\item The idea of cultural relativism is nothing but an excuse to violate human rights.
\end{enumerate}

%\newpage
%\newpage
{\textbf{\Large{About the author}}}
\vspace*{1cm}

Beehnaz Changizi was born on March 21st, 1979 in Hamedan, Iran. She completed
her bachelor studies in Computer Engineering at the Faculty of Computer Engineering Amirkabir University of Technology - Tehran Polytechnic Tehran, Iran, in 2003. She  has worked for several years as a software developer before starting a master's degree. She obtained her master of science in Software Engineering from Sharif University of Technology in 2007.
In 2008, Behnaz moved to become as a Ph.D. student at the
Leiden University as part of the COMPAS project, under the supervision of Prof. Dr. Farhard Arbab. After four years of being a full-time Ph.D. student, Behnaz returned to the industry to follow her passion for creating software, while she kept working on her thesis.

\end{document}
