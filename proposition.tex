\begin{center}{\large{Propositions}}\end{center}
\begin{center}accompanying the thesis\end{center}

\vspace*{.5cm}
\begin{center}{\Large{Constraint-Based Analysis of Business Process Models}}\end{center}

\begin{center}Behnaz Changizi\end{center}
\vspace*{1cm}
\begin{enumerate}
\item The constraint-based nature of our approach
allows simultaneous coexistence of several semantics in a simple fashion, Chapter 1.
%\item BPMN bridges the gap between visualization of the business
%processes and their actual implementation by providing an understandable notation
%for both business stakeholders and technical experts. [Chapter 2]
%\item Writing and maintaining glue code is a tedious task, especially in complex
%systems wherein the size and rigidity of the glue code tend to increase over
%time. This makes these systems hard to modify and maintain. [Chapter 3]
%\item A benefit of employing coordination languages in general and Reo in particular
%is that they express the coordination patterns explicitly and separate them
%from the computational part of the code. [Chapter 4] 
\item By using ATL we benefit from the power of separation of concerns and focus
only on the required mapping rules, rather than matching patterns on the source
models and execution of the rules, Chapter 5.
\item Our framework eliminates the result of expressiveness gap among Reo formal
semantics by incorporating more than one semantics in deriving the behavior
of a Reo connector, Chapter 6.
\item Rather than implementing the highly time- and memory-demanding
custom-made algorithms to generate Reo formal semantics, we use the efficient
SAT-solvers and computer algebra systems to solve constraints whose solutions
are equivalent to these models, Chapter 6.
\item When an end with innate
priority connects to another end that has no priority, the new end will obtain
acquired priority, Chapter 7.
\item Each of these formal semantics can be viewed as a means to constrain the
range of possible behaviors that is expressed in terms of I/O operations through
the nodes to those of allowed by the semantics, Chapter 8.
\item The beauty of mathematics only shows itself to more patient followers.
\item Every culture has something to be ashamed of, but every culture also has the right to change, to challenge negative traditions, and to create new ones.
\item I find that through the study of women, you get to the heart - the truth - of the culture.
\item The idea of cultural relativism is nothing but an excuse to violate human rights.
\end{enumerate}
